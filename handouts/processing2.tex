\documentclass[10pt,twocolumn]{article}

\usepackage[no-math]{fontspec}
\defaultfontfeatures{Scale=MatchLowercase,Mapping=tex-text}
\setmainfont{Garamond Premier Pro}
\setsansfont{Consolas}

\usepackage{balance}
\usepackage[margin=.5in]{geometry}
\setlength{\columnsep}{1cm}
\begin{document}

\thispagestyle{empty}

\twocolumn[
\centerline{\LARGE \bf Processing Reference Sheet II}

\bigskip
\hrule
\bigskip
]

\noindent\textbf{\large Basics}

\begin{description}
\item{\texttt{mouseClicked()}}\ \\[.25em]
Invoked each time the mouse button is clicked.

\item{\texttt{keyPressed()}}\ \\[.25em] 
Invoked each time any key is pressed. The key pressed can be accessed
from \texttt{key}.
\end{description}

\bigskip

\noindent\textbf{\large Arrays}

\begin{description}
\item{\texttt{new A[n]}} \\[.25em]
Creates an new array objects of type \texttt{A} of length
\texttt{n}. 

\item{\texttt{A[] = \{ v1, ... vk \};}} \\[.25em] 
Creates a new array initialized with elements \texttt{v1} through \texttt{vk}. 


\item{\texttt{a.length}} \\[.25em]
The length of array \texttt{a}. 

\item{\texttt{a[i]}} \\[.25em]
The \texttt{i}th element of array \texttt{a}. 

\item{\texttt{a[i] = v}} \\[.25em]
Stores \texttt{v} as the \texttt{i}th element of array \texttt{a}. 

\item{\texttt{for(int i = 0; i < a.length; i++) \{ ... \}}} \\[.25em]
  Iterates through the elements of array \texttt{a}. In each iteration
  of the loop, the element can be accessed using index \texttt{i}.
\end{description}

\bigskip

\noindent\textbf{\large Maps}
\begin{description}
\item{\texttt{google\_map()}}\ \\[.25em]
%
Creates a new Google map object.

\item{\texttt{ms\_map()}}\ \\[.25em]
Creates a new Microsoft map object.

\item{\texttt{map.draw()}}\ \\[.25em] 
Draws map \texttt{map} on the screen.

\item{\texttt{zoom\_ui(map)}}\ \\[.25em] 
Adds handlers to enable zooming and panning using the mouse.

\item{\texttt{location(str)}}\ \\[.25em] 
Computes the geographic coordinates (latitude and longitude) from a
string \texttt{str} containing a USGS site identifier.

\item{\texttt{marker(loc)}}\ \\[.25em] 
Creates a marker from location \texttt{loc}.

\item{\texttt{zoomAndPanTo(loc,lev)}}\ \\[.25em] 
Focuses the map at location \texttt{loc} and zooms in to level \texttt{lev}.

\item{\texttt{map.addMarkers(mark)}}\ \\[.25em] 
Adds marker \texttt{mark} to map \texttt{map}. 

\end{description}

\balance

\bigskip

\noindent\textbf{\large Graph}
\begin{description}

\item{\texttt{Water}}\ \\[.25em] 
An object \texttt{w} of type \texttt{Water} contains the following integer fields. 
\begin{itemize}
\item{\texttt{year}}
\item{\texttt{month}}
\item{\texttt{day}}
\item{\texttt{discharge}}
\item{\texttt{gage}}
\end{itemize}
Each of these fields can be accessed using ``dot'' notation. For
example, to access the \texttt{discharge} field, you would write
\texttt{w.discharge}.

\item{\texttt{parse(str)}}\ \\[.25em] 
Extracts an array of \texttt{Water} objects representing stream flow
data from a string \texttt{str} containing a USGS site identifier.

\item{\texttt{sort(arr)}}\ \\[.25em] 
Sorts the array \texttt{arr} in place. 

\item{\texttt{plot(pairs,bool)}}\ \\[.25em] Creates a chart containing
  a plot of the x-y coordinates represented in array \texttt{pairs};
  uses a semi-logarithmic plot if \texttt{bool} is \texttt{true}.

\item{\texttt{draw\_charts()}}\ \\[.25em] 
Draws all charts on the screen.
\end{description}

\end{document}
