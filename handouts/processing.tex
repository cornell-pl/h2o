\documentclass[10pt,twocolumn]{article}

\usepackage[no-math]{fontspec}
\defaultfontfeatures{Scale=MatchLowercase,Mapping=tex-text}
\setmainfont{Garamond Premier Pro}
\setsansfont{Consolas}

\usepackage{balance}

\usepackage[margin=.5in]{geometry}
\setlength{\columnsep}{1cm}
\begin{document}

\thispagestyle{empty}

\twocolumn[
\centerline{\LARGE \bf Drops Reference Sheet}

\medskip
\hrule
\medskip
]

\noindent\textbf{\large Drop}
\begin{description}
\item{\texttt{d.x}}\ \\[.25em]
%
  The $x$ coordinate of \texttt{d} on the canvas. 

\item{\texttt{d.y}}\ \\[.25em]
%
  The $y$ coordinate of \texttt{d} on the canvas. 

\item{\texttt{d.speed}}\ \\[.25em]
%
  The speed (in the $y$ direction) of \texttt{d} in pixels per frame. 

\item{\texttt{d.dirty}}\ \\[.25em]
%
  A boolean that is \texttt{true} if \texttt{d} is a dirty drop and \texttt{false} otherwise.

\item{\texttt{d.draw()}}\ \\[.25em]
%
  Draw \texttt{d} at its current $x$ and $y$ position.

\item{\texttt{d.caught()}}\ \\[.25em]
%
  Returns \texttt{true} if \texttt{d} has been caught in the bucket.

\item{\texttt{d.splatted()}}\ \\[.25em]
%
  Returns \texttt{true} if \texttt{d} is about to hit the floor.

\item{\texttt{d.dropped()}}\ \\[.25em]
%
  Returns \texttt{true} if \texttt{d} has hit the floor.

\end{description}

\medskip

\noindent\textbf{\large Conditionals and Return}
\begin{description}

\item{\texttt{if(b) \{ ... \} else \{ ... \}}}\ \\[.25em] 
% 
Conditionally executes the first block of code if \texttt{b} evaluates to
\texttt{true}, and the second block of code if it evaluates to \texttt{false}.

\item{\texttt{return b}} \ \\[.25em]
%
  Returns \texttt{b} as the value of the current method.
\end{description}

\medskip

\balance

\noindent\textbf{\large Assignment and Arithmetic}
\begin{description}

\item{\texttt{x = n}} \ \\[.25em]
  Assigns variable \texttt{x} to the value \texttt{v}.

\item{\texttt{n1 + n2}} \ \\[.25em]
  Adds \texttt{n1} to \texttt{n2}. 

\item{\texttt{n1 - n2}} \ \\[.25em]
  Subtracts \texttt{n1} from \texttt{n2}. 

\item{\texttt{n1 * n2}} \ \\[.25em]
  Multiplies \texttt{n1} by \texttt{n2}. 

\item{\texttt{n1 / n2}} \ \\[.25em]
  Divides \texttt{n1} by \texttt{n2} (with no remainder).
\end{description}

\medskip

\noindent{\textbf{Files}
\begin{description}
\item{\texttt{Drops}}\ \\[.25em]
The ``main'' file. Your code goes here.

\item{\texttt{Constants}}\ \\[.25em] 
Defines constants that determine the basic parameters of the game.

\item{\texttt{Globals}}\ \\[.25em] 
Declares global variables that store the game state.

\item{\texttt{Draw}}\ \\[.25em] 
Defines the main \texttt{draw} method, which is invoked every time the screen refreshes.

\item{\texttt{Drop}}\ \\[.25em] 
Defines the \texttt{Drop} data structure.

\item{\texttt{Setup}}\ \\[.25em] 
Defines the main \texttt{setup} method, which is invoked once when the application begins.

\end{description}

\medskip

\noindent\textbf{Advanced Graphics}
\begin{description}
\item{\texttt{background(c)}}\ \\[.25em]
%
Sets the background color to \texttt{c}. There are many ways to describe colors: 
\begin{itemize}
\item \texttt{n}: a single number between \texttt{0} (black) and \texttt{255} (white) describes a shade of grey. 
\item \texttt{(r,g,b)}: a triple of numbers each ranging from
  \texttt{0} to \texttt{255}, describe a color in terms of the
  relative amounts of red, green, and blue.
\end{itemize}

\item{\texttt{fill(c)}}\ \\[.25em]
Sets the current fill color to \texttt{c}. 

\item{\texttt{ellipse(x,y,w,h)}}\ \\[.25em]
Draws an ellipse at coordinates \texttt{x} and \texttt{y} with witdth \texttt{w} and height \texttt{h}. 

\item{\texttt{rect(x,y,w,h)}}\ \\[.25em]
Draws a rectangle at coordinates \texttt{x} and \texttt{y} with witdth \texttt{w} and height \texttt{h}. 

\item{\texttt{mouseX}}\ \\[.25em]
%
The current $x$ coordinate of the mouse pointer.

\item{\texttt{mouseY}}\ \\[.25em]
%
The current $y$ coordinate of the mouse pointer.

\item{\texttt{mouseClicked()}}\ \\[.25em]
Invoked each time the mouse button is clicked.

\item{\texttt{keyPressed()}}\ \\[.25em] 
Invoked each time any key is pressed. The key pressed can be accessed
from \texttt{key}.
\end{description}

\end{document}
