\documentclass[10pt,twocolumn]{article}

\usepackage[no-math]{fontspec}
\defaultfontfeatures{Scale=MatchLowercase,Mapping=tex-text}
\setmainfont{Garamond Premier Pro}
\setsansfont{Consolas}

\usepackage[margin=.5in]{geometry}
\setlength{\columnsep}{1cm}
\begin{document}

\thispagestyle{empty}

\twocolumn[
\centerline{\LARGE \bf Processing Reference Sheet}

\bigskip
\hrule
\bigskip
]

\noindent\textbf{\large Basics}
\begin{description}
\item{\texttt{setup()}}\ \\[.25em]
%
  Invoked once at the start of the execution of the application to
  initialize the screen and set global variables.

\item{\texttt{draw()}} \ \\[.25em]
%
  Invoked repeatedly throughout the execution of the application to
  re-draw the screen.

\item{\texttt{size(width,height)}} \ \\[.25em]
  Sets the size of the screen to \texttt{width} and \texttt{height}
  pixels.

\item{\texttt{millis()}} \ \\[.25em]
%
  Returns the number of milliseconds that have elapsed since the start
  of the execution of the application.
\item{\texttt{mouseX}} \ \\[.25em]
%
  Returns the current x-coordinate of the mouse.

\item{\texttt{mouseY}} \ \\[.25em]
%
  Returns the current y-coordinate of the mouse.
\end{description}

\bigskip

\noindent\textbf{Graphics}
\begin{description}
\item{\texttt{background(c)}}\ \\[.25em]
%
Sets the background color to \texttt{c}. There are many ways to describe colors: 
\begin{itemize}
\item \texttt{n}: a single number between \texttt{0} (black) and \texttt{255} (white) describes a shade of grey. 
\item \texttt{(r,g,b)}: a triple of numbers each ranging from
  \texttt{0} to \texttt{255}, describe a color in terms of the
  relative amounts of red, green, and blue.
\end{itemize}

\item{\texttt{fill(c)}}\ \\[.25em]
Sets the current fill color to \texttt{c}. 

\item{\texttt{ellipse(x,y,w,h)}}\ \\[.25em]
Draws an ellipse at coordinates \texttt{x} and \texttt{y} with witdth \texttt{w} and height \texttt{h}. 

\item{\texttt{rect(x,y,w,h)}}\ \\[.25em]
Draws a rectangle at coordinates \texttt{x} and \texttt{y} with witdth \texttt{w} and height \texttt{h}. 
\end{description}

\bigskip

\noindent\textbf{\large Conditionals and Loops}
\begin{description}

\item{\texttt{if(b) \{ ... \} else \{ ... \}}}\ \\[.25em] 
% 
Conditionally executes the first block of code if \texttt{b} evaluates to
\texttt{true}, and the second block of code if it evaluates to \texttt{false}.

\item{\texttt{for(Object o : drops.toArray()) \\ \{ Drop d = (Drop) o; ... \}}} \ \\[.25em]
%
  Iterates over each drop \texttt{d} in \texttt{drops}. Within
  the block of code \texttt{\{...\}}, the variable
  \texttt{d} refers to the current drop.
\end{description}

\bigskip

\noindent\textbf{\large Assignment and Arithmetic}
\begin{description}

\item{\texttt{x = n}} \ \\[.25em]
  Assigns variable \texttt{x} to the value \texttt{v}.

\item{\texttt{n1 + n2}} \ \\[.25em]
  Adds \texttt{n1} to \texttt{n2}. 

\item{\texttt{n1 - n2}} \ \\[.25em]
  Subtracts \texttt{n1} from \texttt{n2}. 

\item{\texttt{n1 \% n2}}\ \\[.25em]
  Calculates the remainder after dividing \texttt{n1} by \texttt{n2}. 
\end{description}

\bigskip

\noindent\textbf{\large Drops}
\begin{description}

\item{\texttt{new Drop()}}\ \\[.25em]
% 
Creates a new drop at the top of the screen with a random x-coordinate and speed. 

\item{\texttt{d.x}}\ \\[.25em]
% 
The x-coordinate of drop \texttt{d}.

\item{\texttt{d.y}}\ \\[.25em]
% 
The x-coordinate of drop \texttt{d}.
\item{\texttt{d.speed}}\ \\[.25em]
% 
The speed of drop \texttt{d}.

\item{\texttt{d.caught()}}\ \\[.25em]
% 
Returns \texttt{true} if \texttt{d} has been caught by the bucket, and
\texttt{false} otherwise.

\item{\texttt{d.dropped()}}\ \\[.25em]
% 
Returns \texttt{true} if \texttt{d} has been dropped, and
\texttt{false} otherwise.

\item{\texttt{drops.add(d)}}\ \\[.25em]
% 
Add \texttt{d} to the set \texttt{drops}.

\item{\texttt{drops.remove(d)}}\ \\[.25em]
% 
Remove \texttt{d} from the set \texttt{drops}.

\end{description}

\bigskip

\noindent\textbf{\large Helpers}
\begin{description}

\item{\texttt{draw\_drop(d)}}\ \\[.25em] 
%
Draws drop \texttt{d} on the screen. 

\item{\texttt{draw\_bucket(x)}}\ \\[.25em] 
% 
Draws a bucket image at \texttt{x} pixels to the right of the left
edge of the screen.

\item{\texttt{draw\_lives()}}\ \\[.25em] 
% 
Draws the current number of remaining ``lives.''

\item{\texttt{draw\_score()}}\ \\[.25em] 
% 
Draws the current score.

\item{\texttt{draw\_game\_over()}}\ \\[.25em] 
%
Draws a ``game over'' screen and prints the final score. 

\end{description}

\end{document}
